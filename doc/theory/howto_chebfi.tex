\documentclass[11pt]{article}
\usepackage[utf8]{inputenc}
\usepackage[T1]{fontenc}
\usepackage{fullpage}
\usepackage{lmodern}

\usepackage{amsmath,amsfonts,amssymb}

\usepackage{hyperref}

\title{Chebyshev filtering for Abinit users}
\author{Antoine Levitt}
\date{\today}
\begin{document}
\maketitle
\tableofcontents
\section{When to use and not to use Chebfi}
Chebyshev filtering (Chebfi) is a method to solve the linear
eigenvalue problem, and can be used as a SCF solver in Abinit. It was
proposed for use in DFT by Zhou et al. \cite{zhou2006self}, and
implemented in Abinit by Levitt and Torrent \cite{levitt2014parallel} 
(also available in arXiv:1406.4350).

The design goal is for Chebfi to replace LOBPCG as the solver of
choice for large-scale computations in Abinit. By performing less
orthogonalizations and diagonalizations than LOBPCG, scaling to higher
processor counts is possible (see the experiments in
\cite{levitt2014parallel}).

\section{How to use Chebfi}
Simply set \texttt{wfoptalg} to 1, and set the \texttt{np*} variables,
as in LOBPCG. In particular, begin by setting \texttt{npkpt} and
\texttt{npspinor} to the maximum value possible: the tasks are mostly
independent and the speedup nearly optimal. As a starting point, for
large processor counts, use \texttt{npfft} $\approx$ \texttt{npband},
and \texttt{np\_slk} $=$ \texttt{nband}/10.
\section{Convergence}
Mostly due to the absence of a preconditioner in Chebfi, the
convergence is sometimes worse than with LOBPCG. In some cases the
difference is unnoticeable, in others it might be uncompetitively
slow: try for yourself on your system! When convergence is poor, it is
usually a good idea to use more bands than strictly necessary, by
increasing \texttt{nband}. This increases the cost per iteration but
improves convergence: a trade-off is needed. Note that the last bands
will always converge very slowly or not at all, by design: use
\texttt{nbdbuf} to discard these when computing the wavefunction
residual.
\section{Chebfi vs LOBPCG}
Chebfi is usually faster per iteration than LOBPCG, up to factors of
about 5 for high processor counts. It is much less prone to
instabilities and spurious NaNs. It is easier to tune: the options
\texttt{bandpp} and \texttt{use\_slk} are not used, and the
convergence behaviour does not depend on the parallelization - the
results are the same than with the serial version). On the other hand,
for some systems, Chebfi might be very slow to converge.
\section{Performance}
\subsection{Environment}
For good performance it is \emph{imperative} to use optimized linear
algebra libraries. On Intel systems, MKL is a good choice. On other
systems, use the vendor-provided library, or good free ones such as
OpenBLAS. Never use the BLAS/LAPACK fallback in Abinit.

Any good compiler with a reasonable level of optimization (gfortran or
ifort with O2 or greater) should be fine. You should also have a good
MPI implementation.

For large computations, you need to have ScaLAPACK installed. You
should also use the optional ELPA library to speed up the
diagonalizations.

FFTs are usually faster with the FFTW library: see \texttt{fftalg}.
\subsection{Measures}
Use the \texttt{timopt} variable to print a breakdown of time spent at
the end of the output file. If available, PAPI (\texttt{papiopt})
might give more precise measurments. You are looking for the bit that
looks like
\begin{verbatim}
 Partitioning of chebfi
- chebfi                        45.939  34.0     48.072  29.0             48
 
- chebfi(getghc)                27.306  20.2     28.015  16.9            240
- chebfi(opernla)                3.505   2.6      3.583   2.2            192
- chebfi(opernlb)                3.055   2.3      3.124   1.9            192
- chebfi(inv_s)                  2.218   1.6      2.273   1.4            192
- chebfi(alltoall)               0.261   0.2      0.260   0.2             96
- chebfi(rotation)               2.089   1.5      2.146   1.3             48
- chebfi(subdiago)               1.552   1.2      2.596   1.6             48
- chebfi(subham)                 1.000   0.7      1.024   0.6             48
- chebfi(residuals)              0.239   0.2      0.242   0.1             48
- chebfi(update_eigens)          0.226   0.2      0.220   0.1             48
- chebfi(sync)                   3.045   2.3      3.068   1.9             96
\end{verbatim}

The interesting part is the third column, that gives the percent of
total time spent in specific routines. The code scales well (is not
limited by communications) when \texttt{getghc},\texttt{opernla},
\texttt{opernlb} and \texttt{inv\_s} dominate. It has stopped scaling
when the communications dominate : \texttt{alltoall},
\texttt{subdiago} and \texttt{subham}. A large \texttt{sync} is
usually the sign of a suboptimal MPI implementation.

Also interesting is the breakdown of \texttt{getghc}, that gives the
time spent in Fourier and nonlocal operator applications. 
\begin{verbatim}
 Partitioning of getghc
- getghc                        26.335  19.5     27.025  16.3            240
 
- fourwf%getghc                 15.291  11.3     15.692   9.5            240
- nonlop%getghc                 10.849   8.0     11.138   6.7            240
- getghc-other                   0.195   0.1      0.195   0.1            -12
\end{verbatim}

For large computations, the time spent in the FFT (fourwf) operations
should be small.

\section{Tuning and trade-offs}
\subsection{\texttt{nline}}
This option controls the number of applications of the Hamiltonian per
band per iteration. If it is too small, Rayleigh-Ritz steps are too
frequent, which degrades parallel scaling. If it is too large, too
much time is spent optimizing wavefunctions, while the density is not
converged. Don't increase it too much (above 10), or instabilities
will occur.
\subsection{\texttt{np\_slk}}
\texttt{np\_slk} is the number of processors involved in ScaLAPACK
calls. It might be interesting to set this to a lower value than the
total number of processors, as diagonalization stops scaling well
before the rest of the code.

Try varying this and monitoring the \texttt{subdiago} metric. If the
time spent in \texttt{subham} is too high, try reducing
\texttt{np\_slk}. A characteristic value is around \texttt{nband/10},
but your mileage may vary. 
\subsection{\texttt{npfft} vs \texttt{npband}}
For large computations, it is usually a good idea to use as large a
value of \texttt{npfft} as possible (the maximum value is limited by
the size of the FFT grid). If the time spent in \texttt{fourwf} is too
large, decrease \texttt{npfft}.
\subsection{\texttt{use\_gemm\_nonlop}}
This might improve substantially the computation of the nonlocal part
of the Hamiltonian (\texttt{nonlop}). Its usefulness increases with
the number of bands treated by processor, ie
\texttt{nband}/\texttt{npband}.
\subsection{Memory}
Some arrays are distributed on \texttt{npfft} processors only ; some
are distributed on \texttt{np\_slk} only. If you run out of memory,
increasing one of these variables might solve the problem.

\begin{thebibliography}{ZSTC06}

\bibitem[LT14]{levitt2014parallel}
Antoine Levitt and Marc Torrent.
\newblock Parallel eigensolvers in plane-wave density functional theory.
\newblock {\em Comp. Phys. Comm.}, 187:98--105, 2015.

\bibitem[ZSTC06]{zhou2006self}
Yunkai Zhou, Yousef Saad, Murilo~L Tiago, and James~R Chelikowsky.
\newblock Self-consistent-field calculations using chebyshev-filtered subspace
  iteration.
\newblock {\em Journal of Computational Physics}, 219(1):172--184, 2006.

\end{thebibliography}
% \bibliographystyle{alpha}
% \bibliography{refs}
\end{document}
